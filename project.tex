\documentclass[conference]{IEEEtran}
\IEEEoverridecommandlockouts
% The preceding line is only needed to identify funding in the first footnote. If that is unneeded, please comment it out.
\usepackage{cite}
\usepackage{amsmath,amssymb,amsfonts}
\usepackage{algorithmic}
\usepackage{graphicx}
\usepackage{textcomp}
\usepackage{xcolor}
\usepackage[czech]{babel}
\def\BibTeX{{\rm B\kern-.05em{\sc i\kern-.025em b}\kern-.08em
    T\kern-.1667em\lower.7ex\hbox{E}\kern-.125emX}}

\def\abstractname{Abstrakt}
\def\IEEEkeywordsname{Klíčová slova}
\def\refname{Reference}
    
\begin{document}

\title{XSS \-- Cross\--site scripting}

\author{\IEEEauthorblockN{Michal Novák}
\IEEEauthorblockA{\textit{Fakulta informačních technologií} \\
\textit{Vysoké učení technické v~Brně}\\
Brno \\
xnovak3g@stud.fit.vut.cz}}
\maketitle

\begin{abstract}

\end{abstract}

\begin{IEEEkeywords}
XSS, DOM, JavaScript, bezpečnost, zranitelnost
\end{IEEEkeywords}

\section{Úvod XSS}
XSS, neboli Cross-site scripting, je zranitelnost, kterou řadíme mezi takzvané injekční útoky (Injection attacks). ....dodat obecné info

\section{Historie}
XSS lze datovat již do počátku systému World Wide Web (WWW, Web). První webové stránky byly statické a~tvořené čistě jazykem HTML (Hypertext Markup Language). Roku 1995 vzniká programovací jazyk JavaScript, který umožňuje spouštět kód v~prohlížeči klienta a~interagovat s~HTML pomocí objektového modelu dokumentu (DOM,  Document Object Model). Vznikají tak dynamické webové stránky umožňující vytvářet dynamická menu, vyskakovací okna, posuvnou galerii obrázků a~mnoho dalších, dnes již známých prvků moderních uživatelských rozhraní. 

Integrace JavaScriptu do prohlížečů ovšem mimo jiné otevřela vrátka i pro mnohé nekalé praktiky. Útočníci brzy odhalili, že pomocí JavaScriptu mohou velmi jednoduše získat data nic netušících uživatelů. 

Před tím, než vůbec samotné XSS vzniklo, se používaly mnohem jednodušší útoky. Útočníkovi stačilo vytvořit stránku, která v~sobě mohla načíst jinou libovolnou stránku do HTML rámce v~témž okně prohlížeče. Útočníkova stránka poté mohla pomocí JavaScriptu \uv{překročit} hranici a~komunikovat přímo s~načtenou stránkou ve vytvořeném rámci. Daly se takto získat přihlašovací údaje z~formulářů, cookies či jiná citlivá data. Jako ochrana proti tomuto typu útoku byla tedy implementována ochrana \uv{same-origin policy}, která zabraňuje JavaScriptu v~překračování hranic stránek a~limituje jej pouze na původní stránku. Útočníci ovšem toto viděli jako výzvu a~začali odhalovat důmyslnější způsoby, jak omezení obcházet.



~\cite{Grossman2007}


\section{Jak funguje XSS}


\section{Typy XSS}
\subsection{Reflected XSS}
\subsection{Stored XSS}
\subsection{DOM XSS}

\section{Dopady XSS}
\subsection{Příklady použití XSS}
\section{Jak detekovat XSS zranitelnosti}
\section{Předcházení útoků XSS}
\section{Závěr}



\bibliographystyle{IEEEtran}  
\bibliography{project}  
\end{document}
